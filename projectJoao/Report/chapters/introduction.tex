Air pollution causes harm to humans but also other living organisms and may damage the natural and anthropogenic environment. Breathing polluted air can trigger respiratory problems, such as asthma, aggravate health conditions or even cause cancer, lung diseases, brain and nerve damage \cite{epa}.

The society, environmental agencies and governments demonstrate concern about air quality and are actively promoting a cleaner, healthier air in their regions, for example with the application of legislation. The United States Environmental Protection Agency, as an example, produced the Clear Air Act \cite{epa} to control and regulate the production of pollutants in the atmosphere.

Accurate monitoring and prediction is essential to inform and enforce goals established by local authorities. Its importance to policymakers also extends to economic benefits, as high pollution levels incur in costs, for example, increased health costs \cite{zoe}.

In the community and individual level, high-resolution information about air pollution allows citizens to make better decisions regarding commuting around the city or deciding where to spend their time outdoors. Besides, vulnerable people such as people with respiratory diseases can be warned of future short-term pollution peaks that might affect their condition, allowing them to avoid or prepare to minimise risk.

Usually, air quality is monitored by networks of static air quality monitoring stations operated by local city authorities to comply with the statutory requirements for air quality levels. Although these types of equipment are expensive and have to be calibrated at regular intervals, they are accurate and provide detailed results on suspended particulate and gaseous pollutions. However, this means the quantity is limited, and intra-urban environments cannot be studied. 


The AIRSpeck and RESpeck sensors were developed by the Center of Speckled Computing at The University of Edinburgh. The AIRSpeck sensors have a stationary and a mobile form factor, allowing to be either mounted to street furniture or worn as a belt. They measure air quality, including $PM_1$, $PM_{2.5}$ and $PM_{10}$ and are much smaller and cheaper than conventional air monitoring stations. The RESpeck sensor is a respiratory monitor that is worn as a plaster on the chest and collects breathing information such as breathing rate. More information about these sensors is presented in the Background chapter.

Using the AIRSpeck sensors allow for a greater quantity of sensors to be deployed which also allows a higher spatial resolution study of air pollution phenomena. These form a network of sensors collecting data at different times and places and enable an intra-urban analysis, which is not possible with the few reference air quality monitors in a city.

Previous work in the Sensing Spaces project has also focused on air pollution predictions and its applications to citizens. Both spatial and temporal models have been studied and, online methods have been researched, providing evidence that machine learning is a useful technique for air pollution prediction in general. Also, work to visualise, collect and apply this data in a useful and straightforward manner has been developed. Examples of such applications are web interfaces for visualisation of pollution data, android applications that communicate with the sensors for more accessible data collection and path planning algorithms to improve citizen's paths and reduce their exposure to particulate matter. Previous work is described and analysed in detail in the Literature Survey chapter.

The primary goal of the project is to produce an online spatio-temporal model that uses a combination of data from static sensors and data from mobile sensors to predict $PM_{2.5}$ levels in the future. Such a model should make better predictions, compared to the stationary-only case, due to the greater quantity of data and the ability to adapt the model. However, implementing such a model is challenging due to the sparsity of the mobile data and its unpredictability as to when and where we will get this data.

The model proposed uses a spatial and an online temporal model, Ordinary Kriging and Passive-Aggressive Regressor respectively. These are used to interpolate predictions in space and time. This project uses a well-defined area in Edinburgh where data was collected, and predictions were performed. The produced model is online as it learns over time and adapts to changing circumstances, spatial because it can interpolate to regions where observations did not occur and, lastly, temporal as it is able to make predictions into the future.

Such model needs to be compared to other baselines; therefore the implementations of those baselines and the comparison with the model and analysis of the results is also essential for this project and is detailed in the Methodology chapter. 

The model performs better than the baselines, both established and defined by previous work. Moreover, several error correction techniques were also tested, and they were used to prevent the model from underpredicting and to learn using past errors made continuously. This technique improved the results of the model further.

The Android application that connects to the sensors to collect data was also improved to allow users to collect qualitative data in the form of media. The User Interface was enhanced to include new buttons that integrate into the existing application.

\section{Main Contributions}

An improved User Interface was developed for the AIRSpeck android application to allow media collection by the users including photographs, video, text and audio clips. The application and sensors were tested by ten participants and results were collected, visualised and analysed.

For the model, an external dataset was used, so it was visualised, calibrated and outliers were removed. The temporal prediction model was optimised, and its integration with the spatial interpolation method was implemented. The performance of the model was continuously evaluated and compared to different baselines, both newly created and reported in previous work.
The spatio-temporal model performs predictions in space and time dimensions, and the model performed significantly better than baseline and was further improved with an error correction technique. The online characteristic of the model allows a possible real-time deployment of the system, not only predicting in real time but also adapting and training in real time.

The novel contribution of this work is for the first time an online spatio-temporal model has been created using a combination of an online temporal prediction model and spatial interpolation model to predict $PM_{2.5}$ values in high resolution ($48m \times 48m$). It is also the first time that mobile data is used in the training part of the model, improving the performance of the model, instead of just using mobile data as validation of the algorithm.

\section{Report Structure}

In Chapter 2 the background of air pollution, particulate matter, information about the sensors and online machine learning is introduced.

Chapter 3 describes the literature review performed on previous projects on the Sensing Spaces project and other relevant literature.

Chapter 4 describes the development of an improved user interface for the AIRSpeck mobile app and the results of the tests on study participants.

Chapter 5 gives more detail about the methodology, the model proposed, how it works and the justification for the choices made.

In Chapter 6 the results and discussion about the model's behaviour are explained as well as different performance comparisons with baselines.

Lastly, Chapter 7 is a summary of the report and indicates future improvements and the next phases for Part 2 of the MInf project.
