Air pollution and $PM_{2.5}$ particularly has been shown to have an impact on citizen's lives, especially those that have respiratory conditions such as asthma or chronic obstructive pulmonary disease. Intra-urban and short-term predictions can significantly help to inform and to warn people, guided by data-driven decision making, to avoid polluted areas and particulate matter exposure. The nature of air pollution dispersion is complex and dependent on many factors such as meteorological changes, traffic patterns, the dimension of particulates and topography.

Reference sensors, used by official environmental institutions, are used to produce air quality forecasts; however, the dynamics of intra-urban air pollution cannot be represented with data from a limited amount of expensive and accurate sensors. An alternative to this approach, used in this project, is to use a network of inexpensive stationary sensors distributed throughout the city and mobile sensors to complement that data collection. This approach allows new techniques and methods to model air pollution at a high-resolution level.

Previous work on the Sensing Spaces project focused on online, spatial and temporal models independently and mobile data had never been used to train the models. This work did not allow to have a system capable of responding to changes in the environment nor cope with the sparsity of mobile data temporal and spatially. This project proposes an approach that combines Ordinary Kriging and Passive-Aggressive Regressors to make a coupled online spatio-temporal model. This model can adapt with time and predict in areas where data has not been observed. However, the accuracy depends on the amount of data available, when and where it was collected, and how recent it is.

The datasets used in this project both include data from six stationary sensors and two mobile sensors, the latter not used simultaneously. Both datasets include data from four days in July 2018 and four mobile data collection periods at similar times of day, averaging 53 minutes of data per day.

This project has presented a model capable of predicting PM2.5 concentration in intra-urban environments using both stationary sensor data as well as data collected by mobile sensors. It was designed with a real-time context in mind, by predicting and training in real time. The model performs better than a batch learning algorithm given its ability to adapt as new data comes in.  An improved User Interface that allows media collection within the RESPeck android application was also developed and tested with ten participants. Photographs were the most collected type of media, followed by text comments, video clips and lastly audio samples.
The analysis was made with data from July 2018 collected in Edinburgh by ZP with six stationary sensors and two mobile sensors. The experiments demonstrate that time is a better feature for the temporal model than temperature, humidity and past PM2.5 values and that reducing the data noise using a rolling mean of 60 minute does not improve results. The PAR model used for temporal predictions achieved an MAE of 0.159 and MSE of 0.076 with hyperparameters $C= \num{2e-5}$, $epsilon = 0.5$ and $loss = epsilon\_insensitive$ on static data of dataset A, collected in Edinburgh. The spatio-temporal model achieved an MAE of 2.28 on dataset A (19\% better than baseline) and 1.21 on dataset B. 

Five error correction algorithms (last seen value, mean, median, linear regression and passive-aggressive regressor) were tested as methods to improve the prediction output. The regression models showed viability to improve predictions if enough data is made available, but statistical methods performed better with the data currently available. The best error correction algorithm found has a minimum threshold of 3 data points to activate, uses the median as computation method and uses errors of all cells (global data), which further improved the model to an MAE of 0.948 on dataset B, using dataset A as training data, a further 22\% improvement compared to the model without error correction.


\section{Future Work}

It would be interesting to explore the performance of the model when several mobile sensors are concurrently collecting data in the test area along with the stationary sensors, increasing the data available in all time windows and having more grid cells filled with data, reducing sparsity. It would also be interesting to investigate different time periods such as mornings and nights and the model's ability to react to short term patterns such as peak and off-peak hours and other pollution metrics such as $PM_1$ and $PM_{10}$. Further investigation could attempt to demonstrate a correlation between PM values and use that correlation as a feature to the model proposed.

Future work can also concern the application of the models developed to new datasets and possibly new regions such as a dataset from India or an area of a major city such as London. Data from the PHILAP Project is a good example where useful data will be generated. This data will also have corresponding RESpeck data (breathing data), so a more extensive analysis can be done in the Sensing Spaces project, such as how to make these predictions useful in people's daily lives. Besides, a bigger dataset can further extend the error correction methodology, and new results could be derived from the machine learning regressors to be compared with the statistical metrics used currently. 

Other ways of collecting mobile data could be explored to increase the amount of data in the test area, such as placing sensors on public transportation or distributing them to food delivery cyclists that work in unusual periods such as evenings and late night.

The models presented can be used in several applications that future work could cover such as the implementation of a warning system for people especially affected by air pollution, or the integration of this model with path planning algorithms. Moreover, the system can be used to monitor and track air pollution in real time, empowering emergency responders, such as firefighters, with a tool that could detect incidents and anomalies or the citizen's to make better decisions such as where to spend more time outside.  

Several students will be working on the Sensing Spaces project during summer 2019; therefore their work will also be useful for Part II of the project.
